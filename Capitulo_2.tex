% ************ Chapter 2 ************
\chapter{Contexto} 
\label{cap:2}

\section{Estado da arte}
Existem varias ferramentas para fazer a gestão dos recursos de uma empresa e cada uma com as suas características. Para analisar as opções disponíveis, selecionou-se algumas aplicações ERP de forma a poder comparar os seus recursos com as necessidades da empresa.

\subsection{SAP ERP}
SAP ERP  é um sistema integrado de gestão empresarial (ERP) transacional, produto principal da empresa alemã SAP AG líder no segmento de software corporativos.\cite{Wikipediab}. É uma aplicação composta por módulos, nos quais cada modulo é responsável por uma parte da atividade da empresa.


\subsection{Primavera Executive}
O PRIMAVERA Executive é um software ERP desenvolvido pela empresa PRIMAVERA Business Software Solutions com foco nas pequenas e médias empresas \cite{PRIMAVERABSS}. A PRIMAVERA Business Software Solutions é uma empresa que se dedica ao desenvolvimento e comercialização de soluções de gestão e plataformas para integração de processos empresariais \cite{Wikipediaa}. Esta tecnológica portuguesa que se afirmou no mercado nacional de soluções informáticas de gestão por ser pioneira no desenvolvimento de aplicações para Windows. \cite{Wikipediaa}.
Esta solução inclui ainda integração com os softwares de faturação PRIMAVERA.

\subsection{Microsoft Dynamics}
Microsoft Dynamics é um pacote software da Microsoft destinado a gestão corporativa ERP, para ajudar na tomada de decisões gerenciais e melhorar os resultados administrativos e financeiros das empresas.\cite{Wikipediac}
Dentro deste pacote de software existem as seguintes aplicações:
\begin{itemize}
	\item Dynamics 365 for Sales
	\item Dynamics 365 for Retail
	\item Dynamics 365 for Finance and Operations
	\item Dynamics 365 for Talent
\end{itemize}
Este pacote está disponível em várias edições nas quais varia as funcionalidades disponíveis de modo a se ajustar melhor às necessidades da empresa.

\section{Tabela comparativa}
\begin{table}[H]
	\resizebox{\textwidth}{!}{%
		\begin{tabular}{|l|c|c|c|c|}
			\hline
			Característica & SAP & Primavera Executive & Microsoft Dynamics & Solução Própria\\ \hline
			Funções de gestão de recursos da empresa
										& \ding{51} & \ding{51} & \ding{51} & \ding{51} \\ \hline
			Acesso remoto externo		& \ding{53} & \ding{51} & \ding{51} & \ding{53}\\ \hline
			Suporte técnico de terceiros& \ding{51} & \ding{51} & \ding{51} & \ding{53}\\ \hline
			Controlo do desenvolvimento & \ding{53} & \ding{53} & \ding{53} & \ding{51}\\ \hline
			Atualizações do sistema		& \ding{51} & \ding{51} & \ding{51} & tem de as desenvolver\\ \hline
			Adaptação total do sistema às necessidades da empresa
										& \ding{53} & \ding{53} & \ding{53} & \ding{51}\\ \hline
			Implementação apenas dos recursos necessário
										& \ding{53} & \ding{53} & \ding{53} & \ding{51}\\ \hline
		\end{tabular}%
	}
\end{table}

\section{Opção escolhida}
No final desta analise, o caminho definido para o projeto foi a criação de uma nova plataforma, sendo que essa era a intenção da empresa desde o incio. Por esse motivo foi feita uma nova analise de opções à cerca do desenvolvimento da plataforma. Este documento foi entregue à administração para que o analisa-se o decidisse o modo como a aplicação deveria ser desenvolvida. O documento entregue esta presente no \hyperref[anexo:A]{Anexo A}.