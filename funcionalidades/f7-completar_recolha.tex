\subsection{Aplicação Fábrica: Completar recolha}
\subsubsection*{Descrição do caso de uso}
Para completar recolha, espera-se que utilizador entre na página e indique o ID do ponto de recolha e o peso que pretende acrescentar à recolha. A informação sobre o peso total já registado bem como o histórico de incrementos será apresentado automaticamente pelo sistema. 

\begin{figure}[H] 
	\begin{center}
		\includegraphics[width=0.60\textwidth,keepaspectratio]{figuras/Diagramas_vp/DI_Fabrica_7_Completa_Recolha.jpg}
		\caption{Modelo do formulário para incrementar o peso de uma recolha}
		\label{fig:di_completar_recolha} 
	\end{center}
\end{figure}

\subsubsection*{Fluxo do caso de uso}
O caso de uso inicia-se com a abertura da página do registo de recolha. É apresentado o formulário com a data previamente preenchida. O utilizador tem de indicar o ID do ponto de recolha numa lista de dropdown e o peso da recolha feita. Após indicar as informações solicitadas precisona o botão "Guardar". No final do registo é apresentada uma mensagem ao utilizador. Caso o registo seja feito com sucesso um novo separador é aberto com o código de barras para ser impresso.


\begin{figure}[H] 
	\begin{center}
		\includegraphics[width=\textwidth,keepaspectratio]{figuras/Diagramas_vp/SD_Fabrica_7_Completar_Recolha.jpg}
		\caption{Diagrama de sequência para incrementar o peso de uma recolha}
		\label{fig:sd_completar_recolha} 
	\end{center}
\end{figure}