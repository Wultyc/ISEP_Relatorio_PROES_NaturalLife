% ************ Anexo  ************
%\renewcommand{\appendixname}{Anexo}
%\appendix % Usar este comando só no primeiro Anexo
\chapter{\textit{Stored procedure} fim de protocolo}
\label{anexo:C}
Código utilizado na \textit{stored procedure Fim de Protocolo} para notificar a administração da empresa sobre quais Pontos de Recolha cujo o protocolo está prestes terminar.

\begin{verbatim}
USE [NaturalLife]
GO
/****** Object:  StoredProcedure [dbo].[FimProtocolos]    
Script Date: 16/07/2019 08:38:38 ******/
SET ANSI_NULLS ON
GO
SET QUOTED_IDENTIFIER ON
GO
-- =============================================
-- Author:		NaturalLife
-- Create date: 03-07-2019
-- Description:	Alerta pontos de recolha
-- =============================================
ALTER PROCEDURE [dbo].[FimProtocolos]
AS
BEGIN

--params 1º alerta: 90 dias, 2º alerta: 30 dias, 3º alerta: 15 dias
DECLARE @Alerta1 AS INT = 90 -- 1º Alerta
DECLARE @Alerta2 AS INT = 30 -- 2º Alerta
--DECLARE @Alerta3 AS INT = 0 -- 3º Alerta


IF EXISTS(SELECT
pontos_recolha.id,
pontos_recolha.nome,
pontos_recolha.fimProtocolo,
DATEDIFF(day, GETDATE(), pontos_recolha.fimProtocolo)
FROM pontos_recolha
WHERE   
DATEDIFF(day, GETDATE(), pontos_recolha.fimProtocolo) 
in (@Alerta1, @Alerta2))
BEGIN

--Declara as variaveis
Declare @HTMLBody nvarchar(max),
@tableBody nvarchar(max)

--Cria a tabela HTML
SET @tableBody = CONVERT(NVARCHAR(MAX), (SELECT
(SELECT '' FOR XML PATH(''), TYPE) AS 'caption',
(SELECT 
'ID' AS th,
'Nome' AS th,
'Fim do protocolo' AS th,
'Dias Restantes' AS th 
FOR XML RAW('tr'), ELEMENTS, TYPE) AS 'thead',
(

--Inicio da Query
SELECT
pontos_recolha.id as td,
pontos_recolha.nome as td,
pontos_recolha.fimProtocolo as td,
DATEDIFF(day, GETDATE(), pontos_recolha.fimProtocolo) as td
FROM pontos_recolha
WHERE   
DATEDIFF(day, GETDATE(), pontos_recolha.fimProtocolo) 
in (@Alerta1, @Alerta2)
ORDER BY DATEDIFF(day, GETDATE(), pontos_recolha.fimProtocolo) ASC
--Fim da Query

FOR XML RAW('tr'), ELEMENTS, TYPE
) AS 'tbody'
FOR XML PATH(''), ROOT('table')));


--Corpo do HTML
SET @HTMLBody = '<html><head><style>
table, th, td {
border: 1px solid black;
}
table {
width: 100%;
border-collapse: collapse;
}
th {
width: 25%;
background-color: #99CCFF;
}
tr {
width: 25%;
background-color: #F1F1F1;
}
</style><title>Registo de Produção maior que Registo de Recolha</title>
</head><body>'
--SET @HTMLBody = @HTMLBody + 'Aproxima-se a data do fim de protocolo 
com os seguintes Pontos de Recolha<br/>'
SET @HTMLBody = @HTMLBody + @tableBody + '</body></html>'

--envia o email
exec msdb.dbo.sp_send_dbmail 
@profile_name = 'NaturalLife', 
@recipients = '<Email de Destinatário>',
@subject = '[ALERTA] Fim de Protocolo', 
@body = @HTMLBody, 
@body_format = 'HTML'

END

END

\end{verbatim}

