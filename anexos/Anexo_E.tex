% ************ Anexo  ************
%\renewcommand{\appendixname}{Anexo}
%\appendix % Usar este comando só no primeiro Anexo
\chapter{\textit{Stored procedure} erros no negisto de ponto}
\label{anexo:E}
Código utilizado na \textit{stored procedure Erros no Registo de Ponto} para notificar a administração da empresa sobre quais os registos de ponto não são coerentes.

\begin{verbatim}
USE [NaturalLife]
GO
/****** Object:  StoredProcedure [dbo].[ErroPonto]    
Script Date: 16/07/2019 08:48:45 ******/
SET ANSI_NULLS ON
GO
SET QUOTED_IDENTIFIER ON
GO
-- =============================================
-- Author:		NaturalLife
-- Create date: 03-07-2019
-- Description:	Alerta erros nos registos de ponto
-- =============================================
ALTER PROCEDURE [dbo].[ErroPonto] AS
BEGIN
DECLARE @SQL table(id int NULL, nome nvarchar(max) NULL, data date NULL)

insert into @SQL
-- Número de registos ímpar
SELECT
colaboradores.id AS ID,
colaboradores.nome AS NOME,
cast(ponto.dataHora as date) AS DATA
--colaboradores.nome AS 'Nome',
--COUNT(*) AS 'Contador'
FROM ponto
INNER JOIN colaboradores
ON ponto.colaboradores_id = colaboradores.id
WHERE cast(ponto.dataHora as date)= cast(GETDATE()-1 as date)
GROUP BY colaboradores.id, colaboradores.nome, 
cast(ponto.dataHora as date)
HAVING COUNT(*) % 2 <> 0

union

-- Registos com menos de 10 minutos de diferença
SELECT  P1.colaboradores_id AS 'ID_Colab_RegProx',
colaboradores.nome AS 'Nome Colab',
cast(P1.dataHora as date) AS 'Data1'
-- MIN(P2.dataHora) AS 'Data2',
--DATEDIFF(minute, P1.dataHora, MIN(P2.dataHora)) AS 'Diferença'

FROM ponto as P1
LEFT JOIN ponto P2 ON P1.colaboradores_id = P2.colaboradores_id
AND P2.dataHora > P1.dataHora
INNER JOIN colaboradores
ON P1.colaboradores_id = colaboradores.id
WHERE cast(P1.dataHora as date) = cast(GETDATE()-1 as date)
AND cast(P2.dataHora as date) = cast(GETDATE()-1 as date)
AND DATEDIFF(minute, P1.dataHora, P2.dataHora) < 10
GROUP BY P1.colaboradores_id, P1.dataHora, colaboradores.nome

union

-- Número de registos de entrada diferente de saída
SELECT
colaboradores.id AS ID,
colaboradores.nome AS NOME,
cast(ponto.dataHora as date) AS DATA
--colaboradores.nome AS 'Nome',
--COUNT(*) AS 'Contador'
FROM ponto
INNER JOIN colaboradores
ON ponto.colaboradores_id = colaboradores.id
WHERE cast(ponto.dataHora as date) = cast(GETDATE()-1 as date)
GROUP BY colaboradores.id, colaboradores.nome, cast(ponto.dataHora as date)
HAVING sum(case when entrada = 1 then 1 else 0 end) <> 
sum(case when entrada = 0 then 1 else 0 end)

IF EXISTS(
SELECT * FROM @SQL
)
BEGIN

--Declara as variaveis
Declare @HTMLBody nvarchar(max),
@tableBody nvarchar(max)

--Cria a tabela HTML
SET @tableBody = CONVERT(NVARCHAR(MAX), (SELECT
(SELECT '' FOR XML PATH(''), TYPE) AS 'caption',
(SELECT 
'ID' AS th,
'Nome' AS th,
'Data' AS th,
'Mais Detalhes' AS th 
FOR XML RAW('tr'), ELEMENTS, TYPE) AS 'thead',
(

--Inicio da Query
SELECT 
id as td,
nome as td,
data as td,
cast('<a href =
 "http://192.168.1.67/Painel/Colaboradores/Ponto/Erro?SearchOnURL=' + 
 cast(id as nvarchar) + '_' +  cast(cast(data as date) as nvarchar) +
  '"  target="_blank"> Ver no Painel </a>' as XML) as td

FROM @SQL
--Fim da Query

FOR XML RAW('tr'), ELEMENTS, TYPE
) AS 'tbody'
FOR XML PATH(''), ROOT('table')));


--Corpo do HTML
SET @HTMLBody = '<html><head><style>
table, th, td {
border: 1px solid black;
}
table {
width: 100%;
border-collapse: collapse;
}
th {
width: 25%;
background-color: #99CCFF;
}
tr {
width: 25%;
background-color: #F1F1F1;
}
</style><title>Registo de Produção maior que Registo de Recolha</title>
</head><body>'
--SET @HTMLBody = @HTMLBody + 'Aproxima-se a data do fim de protocolo
 com os seguintes Pontos de Recolha<br/>'
SET @HTMLBody = @HTMLBody + @tableBody + '</body></html>'

--envia o email
exec msdb.dbo.sp_send_dbmail 
@profile_name = 'NaturalLife', 
@recipients = '<Email de Destinatário>',
@subject = '[ALERTA] Erros de Ponto', 
@body = @HTMLBody, 
@body_format = 'HTML'
END
END
\end{verbatim}