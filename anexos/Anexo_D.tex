% ************ Anexo  ************
%\renewcommand{\appendixname}{Anexo}
%\appendix % Usar este comando só no primeiro Anexo
\chapter{\textit{Stored procedure} erros de produção}
\label{anexo:D}
Código utilizado na \textit{stored procedure Erros de Produção} para notificar a administração da empresa sobre quais os registos de produção têm um peso superior ao peso da recolha associada.

\begin{verbatim}
USE [NaturalLife]
GO
/****** Object:  StoredProcedure [dbo].[ErrosProducao]    
Script Date: 16/07/2019 08:47:51 ******/
SET ANSI_NULLS ON
GO
SET QUOTED_IDENTIFIER ON
GO
-- =============================================
-- Author:		NaturalLife
-- Create date: 03-07-2019
-- Description:	Alerta erros de produção
-- =============================================
ALTER PROCEDURE [dbo].[ErrosProducao]
AS
BEGIN

IF EXISTS(SELECT
recolhas.id,
recolhas.data_recolha,
recolhas.peso,
SUM(producao.peso_cera + producao.peso_chapa + producao.peso_plastico)
FROM producao
INNER JOIN recolhas
ON	recolhas.id = producao.recolhas_id
WHERE (producao.peso_cera + producao.peso_chapa + producao.peso_plastico) > recolhas.peso
AND producao.created_at > GETDATE()-1
GROUP BY recolhas.id, recolhas.data_recolha, recolhas.peso)
BEGIN
--Declara as variaveis
DECLARE @HTMLBody NVARCHAR(MAX),
@tableBody NVARCHAR(MAX)

--Cria a tabela HTML
SET @tableBody = CONVERT(NVARCHAR(MAX), (SELECT
(SELECT '' FOR XML PATH(''), TYPE) AS 'caption',
(SELECT 
'ID Recolha' AS th,
'Data de Recolha' AS th,
'Peso de Recolha' AS th,
'Peso de Produção' AS th 
FOR XML RAW('tr'), ELEMENTS, TYPE) AS 'thead',
(

--Inicio da Query
SELECT
recolhas.id AS td,
recolhas.data_recolha AS td,
recolhas.peso AS td,
SUM(producao.peso_cera + producao.peso_chapa + producao.peso_plastico) AS td
FROM producao
INNER JOIN recolhas
ON	recolhas.id = producao.recolhas_id
WHERE (producao.peso_cera + producao.peso_chapa +
 producao.peso_plastico) > recolhas.peso
AND producao.created_at > GETDATE()-1
GROUP BY recolhas.id, recolhas.data_recolha, recolhas.peso
ORDER BY recolhas.id DESC
--Fim da Query

FOR XML RAW('tr'), ELEMENTS, TYPE
) AS 'tbody'
FOR XML PATH(''), ROOT('table')));


--Corpo do HTML
SET @HTMLBody = '<html><head><style>
table, th, td {
border: 1px solid black;
}
table {
width: 100%;
border-collapse: collapse;
}
th {
width: 25%;
background-color: #99CCFF;
}
tr {
width: 25%;
background-color: #F1F1F1;
}
</style><title>Registo de Produção maior que Registo de Recolha</title>
</head><body>'
--SET @HTMLBody = @HTMLBody + 'Os seguintes registos de produção
 superaram o peso da recolha associada<br/>'
SET @HTMLBody = @HTMLBody + @tableBody + '</body></html>'

--envia o email
EXEC msdb.dbo.sp_send_dbmail 
@profile_name = 'NaturalLife', 
@recipients = '<Email de Destinatário>', 
@subject = '[ALERTA] Registo de Produção maior que Registo de Recolha', 
@body = @HTMLBody, 
@body_format = 'HTML'
END

END

\end{verbatim}