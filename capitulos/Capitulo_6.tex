% ************ Chapter 7 ************
\chapter{Conclusões}
\label{cap:6}
\section{Resumo do relatório}
Este projeto foi executado na empresa Natural Life, Lda. Esta é uma empresa que se dedica à reciclagem de resíduos de velas de cerae que necessitava de um sistema de informação robusto que assegurasse a informação nele armazenada. No âmbito deste projeto foi feito o estudo completo do sistema anterior de forma a recolher o máximo de informação possível para a construção do novo sistema. O novo sistema satisfez bastante os administradores que estes optaram por o manter a execução. Do ponto de vista dos colaboradores a reação foi também bastante positiva.


\section{Objetivos realizados}
A administração da empresa pretendia que fosse implementado um novo sistema de informação para efetuar os de informação proveniente do trabalho da fábrica e que esta pudesse ser acedida e manipulada por uma área restrita. O sistema ficou bem ajustado ao processo da empresa e todos os colaboradores tiveram uma transição suave. Por estes motivos é correto afirmar que os objetivos foram atingidos totalidades.

\section{Limitações e trabalho futuro}
Apesar da aplicação estar muito bem integrada na fábrica neste momento é necessário ter a consciência de que com a evolução da fábrica poderá ser necessário fazer alguma atualização ao sistema. É importante realçar que sendo uma aplicação própria da empresa, as funcionalidades que são implementadas são exatamente o que a empresa necessita, mas fica também a cargo desta criar todas as condições necessárias para manter a estabilidade e seguraça deste sistema.

\section{Apreciação final}
Este foi um projeto muito relevante, não só pela exigência que construir um sistema responsável por armazenar toda a informação da empresa tendo em atenção todos os detalhes de modo a que a aplicação não prejudicasse a empresa, mas também porque se colocar problemas do mundo real, cujo em alguns casos as soluções tinham de ser rápidas.
A apreciação global é positiva tendo em conta não só a prepetiva da empresa que propós o projeto, como do estudante da LES que o aceitou e concretizou ate ao fim.
Por fim 