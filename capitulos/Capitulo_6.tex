% ************ Chapter 7 ************
\chapter{Conclusões}
\label{cap:6}
O capítulo de conclusões é um dos mais importantes do relatório, sendo aqui que devem ser apresentados os resultados do trabalho efetivamente desenvolvido. 
As conclusões finais devem focar o sucesso/insucesso do trabalho, revendo as dificuldades encontradas. Devem resumir, de alguma forma, as vantagens do produto desenvolvido e a utilidade que possa ter para a instituição de estágio ou para os seus clientes/parceiros. Podem também referir a forma como o estágio decorreu, bem como a integração, a formação dada pela instituição, as facilidades e as dificuldades sentidas ao longo do estágio.
As conclusões devem basear-se nos resultados realmente obtidos. Devem enquadrar-se os resultados obtidos com os objetivos enunciados e procurar extrair conclusões mais gerais, eventualmente sugeridas pelos resultados. Podem acompanhar as conclusões incluindo recomendações apropriadas, resultantes do trabalho, nomeadamente sugerindo e justificando eventuais extensões e modificações futuras.


\section{Resumo do relatório}

Esta secção é opcional, servindo apenas para relembrar os pontos mais importantes focados nos capítulos anteriores.


\section{Objetivos realizados}

Nesta secção devem ser repetidos os objetivos apresentados no capítulo de introdução e, para cada um deles, deve ser descrito o seu grau de realização. Recomenda-se o uso de uma lista, dado que facilita a compreensão pelo leitor.

\section{Limitações e trabalho futuro}

Nesta secção devem ser identificados os limites do trabalho realizado (condições de operação), fazendo uma análise autocrítica ao trabalho, bem como extrapolar sobre as possíveis direções de desenvolvimento futuro.

\section{Apreciação final}

Esta secção deve fornecer uma opinião pessoal sobre o trabalho desenvolvido. Nomeadamente o seu contributo para o desenvolvimento pessoal e profissional.