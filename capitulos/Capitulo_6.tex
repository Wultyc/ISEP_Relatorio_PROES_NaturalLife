% ************ Chapter 7 ************
\chapter{Conclusões}
\label{cap:6}
\section{Resumo do relatório}
No inicio deste documento descreveu-se o paradigma de uso e reciclagem de velas de cera e a atuação da empresa Natural Life face a esta situação. De seguida fez-se a analise das limitações da aplicação existente e das soluções que havia atualmente para resolver esse problema. O terceiro capitulo foi dedicado unicamente ao estudo da solução que existia na empresa e dos requisitos do novo sistema. Este foi um passo fundamental para os capítulos da projeção e implementação que se seguiram.


\section{Objetivos realizados}
Quando este projeto foi inciado, a administração da empresa NaturalLife pretendia que fosse implementado um novo sistema de informação para efetuar o registo do ponto dos colaboradores, recolhas, produções, acabamento e saída do produto acabado. Esta solução teria de possuir duas áreas distintas, a primeira destinada ao uso na fábrica e a segunda destinada ao uso pela administração. Esta solução pretendia-se bem integrada no processo da empresa e que levasse o menor leve possível de adaptação ao novo sistema por parte dos funcionários. A solução desenvolvida cumpriu com todos os objetivos existentes.

\section{Limitações e trabalho futuro}
Nesta secção devem ser identificados os limites do trabalho realizado (condições de operação), fazendo uma análise autocrítica ao trabalho, bem como extrapolar sobre as possíveis direções de desenvolvimento futuro.

\section{\textit{Feedback} recebido}
No final do projeto havia um sentimento generalizado de satisfação por toda a empresa. Os colaboradores ficaram muito agradados com simplicidade e coesão gráfica do sistema. Foi muito fácil para todos eles se adaptarem ao sistema que foi desenvolvido. Da parte da administração este também ficaram muito satisfeitos pela intuitividade da aplicação e da facilidade com que tinham acesso à informação sem necessidade de se deslocarem até ao computador na fábrica. De todos os comentários o mais direto partiu diretamente do Engº Telmo Azevedo que demonstrou-se bastante satisfeito com o resultado final, apesar deste sentir que o resultado final não era exatamente como ele espera, conseguia perceber o acréscimo de valor que o sistema trazia para a empresa. Na sua opinião o processo de adaptação que ele teria de sofrer era largamente compensado pela agilidade que o novo sistema trouxe para a empresa.

\section{Apreciação final}
Do meu ponto de vista pessoal, foi um projeto que me acrescentou muito valor. A exigência de lidar com o desenvolvimento do sistema responsável por armazenar toda a informação da empresa 